\chapter{Príloha B}
\label{pri:priloha2}
\section{Krivky druhého stupňa}
Krivka druhého stupňa $c$ je v karteziánskych súradniciach $(x, y) \in \mathbb{E}^2$ daná rovnicou 
$$ q(x, y) = Ax^2 + Bxy + Cy^2 + Dx + Ey + F = 0.$$
Prípad $A = B = C = 0$  vylúčime, pretože potom je rovnica lineárna a opisuje priamku. Vo všeobecnosti rovnica opisuje kužeľosečku, ktorá je daná piatimi bodmi. Ak $F \neq 0$, môžeme ju vydeliť $F$ a potom riešiť sústavu lineárnych rovníc s piatimi neznámymi. Okrem klasických prípadov, ako elipsy, paraboly a hyperboly, môžu kužeľosečky degenerovať na dvojice priamok, bod alebo prázdnu množinu.

V algebraickom zmysle má kužeľosečka $c$ vždy dva priesečníky $S_1$ a $S_2$ s danou priamkou $s$. Oba môžu byť reálne alebo komplexne združené. Limitný prípad $S_1 = S_2$ nastáva vtedy, keď $s$ je dotyčnicou k $c$.

V závislosti od počtu reálnych priesečníkov s priamkou $s$ v nekonečne rozlišujeme tri typy kužeľosečiek 
\begin{enumerate}
\item eliptický typ: bez reálnych priesečníkov,
\item hyperbolický typ: dva reálne priesečníky,
\item parabolický typ: priamka v nekonečne sa dotýka krivky. \cite{Gla16}\cite{Ode20}
\end{enumerate}
Ku klasifikácii kužeľosečiek nahliadnime do lineárnej algebry a geometrie \cite{Kor13} a \cite{Zla11}.

Matica $A \in \mathbb{R}^{n \times n}$ sa nazýva ortogonálna, ak platí $A^T \cdot A = I_n$, alebo, čo je to isté, $A^{-1} = A^T$. Uvedomme si, že prvá podmienka vlastne hovorí, že stĺpce matice $A$ tvoria ortonormálnu bázu euklidovského priestoru $\mathbb{R}^n$ so štandardným skalárnym súčinom. Potom tiež platí $A \cdot A^T = I_n$, teda takisto riadky matice $A$ tvoria ortonormálnu bázu v $\mathbb{R}^n$. 

\begin{theorem} 
Matica prechodu od ortonormálnej bázy v $\mathbb{R}^n$ so štandardným skalárnym súčinom k ortonormálnej báze je ortogonálna matica. Tiež, ak od ortonormálnej bázy v $\mathbb{R}^n$ prejdeme pomocou ortogonálnej matice prechodu k novej báze,
tak aj nová báza bude ortonormálna.
\end{theorem}

\subsection{Invarianty kriviek druhého stupňa}
\begin{definition}
Invariantom krivky druhého stupňa, vyjadrenej rovnicou
$$
q(x, y) = a_{11}x^2 + 2a_{12}xy + a_{22}y^2 + 2a_{13}x + 2a_{23}y + a_{33} = 0
$$
je každý taký algebraický výraz, závisiaci od \(a_{11}, a_{12}, a_{22}, a_{13}, a_{23}, a_{33}\), ktorého hodnota sa nezmení, ak túto krivku vyjadríme v inom karteziánskom súradnicovom systéme, ku ktorému prejdeme pomocou otočení alebo posunutí (čím od rovnice, viažúcej staré premenné \(x, y\), prejdeme k rovnici, viažúcej nové premenné \(x', y'\)).
\end{definition}

\begin{theorem}
Nasledujúce číselné výrazy sú invariantmi krivky druhého stupňa, vyjadrenej rovnicou $q(x, y)$.
\begin{align*}
\Delta(x,y) &= \det \begin{pmatrix} 
a_{11} & a_{12} & a_{13} \\ 
a_{12} & a_{22} & a_{23} \\
a_{13} & a_{23} & a_{33} \end{pmatrix}, \
\delta(x,y) = \det \begin{pmatrix} a_{11} & a_{12} \\ a_{12} & a_{22} \end{pmatrix}, \
s(x,y) &= a_{11} + a_{22}.
\end{align*}
\end{theorem}
Z tohto možno odvodiť nasledujúcu klasifikáciu kužeľosečiek.

%\begin{table}[h]
%\centering
%\begin{tabular}{|c|c|c|l|}
%\hline
%\textbf{Typ} & $\delta$ & $\Delta$ & \textbf{Tvar}  \\
%\hline
%\multirow{3}{*}{eliptický} & \multirow{3}{*}{$> 0$} & $\neq 0$ & ak $s\Delta < 0$, tak elipsa \\
%& & $\neq 0$ & ak $s\Delta > 0$, tak  $\emptyset$ \\
%& & $= 0$ & bod \\
%\hline
%\multirow{2}{*}{hyperbolický} & \multirow{2}{*}{$< 0$} & $\neq0$ & hyperbola \\
% & & $=0$ & dve rôznobežné priamky \\
%\hline
%\multirow{4}{*}{parabolický} & \multirow{4}{*}{$= 0$} & $\neq0$ & parabola \\
%& & $=0$ & dve rovnobežné priamky \\
%& & $= 0$ & priamka \\
%& & $= 0$ & $\emptyset$ \\
%\hline
%\end{tabular}
%\caption{Klasifikácia kužeľosečiek.}
%\label{tab:conic_sections}
%\end{table}

\begin{table}[h]
\centering
\begin{tabular}{|l|l|l|l|}
\hline
\textbf{Typ} & $\delta$ & $\Delta \neq  0$ & $\Delta = 0 $ \\
\hline
\multirow{2}{*}{eliptický} & \multirow{2}{*}{$> 0$} & ak $s\Delta < 0$, tak elipsa & \multirow{2}{*}{bod} \\
& & ak $s\Delta > 0$, tak $\emptyset$ & \\
\hline
hyperbolický & $< 0$ & hyperbola & dve rôznobežné priamky \\
\hline
\multirow{3}{*}{parabolický} & \multirow{3}{*}{$= 0$} & dve rovnobežné priamky & \multirow{3}{*}{parabola} \\
& & priamka & \\
& & $\emptyset$ & \\
\hline
\end{tabular}
\caption{Klasifikácia kužeľosečiek.}
\label{tab:conic_sections}
\end{table}

\section{Plochy druhého stupňa}
Plocha druhého stupňa $Q$ je v karteziánskych súradniciach $(x, y, z) \in \mathbb{E}^3$ daná rovnicou
\[ q(x, y, z) = Ax^2 + By^2 + Cz^2 + Dxy + Exz + Fyz + Gx + Hy + Iz + J = 0. \]
Je zrejmé, že ak je prvých šesť koeficientov nulových, uvedená rovnica je lineárna a opisuje rovinu v priestore. Vo všeobecnosti rovnica opisuje kvadriku, ktorá je daná deviatimi bodmi. Ak $J \neq 0$, môžeme ju vydeliť $J$ a potom vyriešiť sústavu lineárnych rovníc s deviatimi neznámymi. Okrem klasických prípadov, ako elipsoidy, paraboloidy a hyperboloidy, môžu kvadriky degenerovať aj na kvadratické kužele, kvadratické valce a dvojice rovín.

V algebraickom zmysle je kvadrika $Q$ plocha druhého stupňa, ktorá má vždy dva priesečníky $S1$ a $S2$ s danou priamkou $s$. Oba môžu byť reálne alebo komplexne združené. Limitný prípad $S1 = S2$ nastáva vtedy, keď $s$ je dotyčnicou $Q$.

V závislosti od počtu reálnych priesečníkov s rovinou $s$ v nekonečne rozlišujeme tri typy kvadrík:
\begin{enumerate}
\item eliptický typ: bez reálnych priesečníkov,
\item hyperbolický typ: dva reálne priesečníky,
\item parabolický typ: kvadriky sa dotýka rovina v nekonečne. \cite{Ode20}
\end{enumerate}

Nasledujúce číselné výrazy sú invariantmi plochy druhého rádu, vyjadrenej rovnicou 
\[ q(x, y, z) = a_{11}x^2 + a_{22}y^2 + a_{33}z^2 + 2a_{12}xy + 2a_{13}xz + 2a_{23}yz + 2a_{14}x + 2a_{24}y + 2a_{23}z + a_{44} = 0. \]
\begin{align*}
\Delta(x, y, z) &= \det \begin{pmatrix} 
a_{11} & a_{12} & a_{13} & a_{14} \\ 
a_{21} & a_{22} & a_{23} & a_{24} \\
a_{31} & a_{32} & a_{33} & a_{34} \\
a_{41} & a_{42} & a_{43} & a_{44}
\end{pmatrix}, \
\delta(x, y, z) = \det \begin{pmatrix} 
a_{11} & a_{12} & a_{13} \\ 
a_{21} & a_{22} & a_{23} \\ 
a_{31} & a_{32} & a_{33} 
\end{pmatrix}
\end{align*}
\begin{align*}
T(x, y, z) &= \det \begin{pmatrix} 
a_{11} & a_{12} \\ 
a_{21} & a_{22} 
\end{pmatrix} + \det \begin{pmatrix} 
a_{22} & a_{23} \\ 
a_{32} & a_{33} 
\end{pmatrix} + \det \begin{pmatrix} 
a_{33} & a_{31} \\ 
a_{13} & a_{11} 
\end{pmatrix}, 
\end{align*}
\begin{align*}
s(x, y, z) &= a_{11} + a_{22} + a_{33}, \ \Delta'(x, y, z) = \Delta_{11} + \Delta_{22} + \Delta_{33}, 
\end{align*}
kde $
\Delta_{ij} = (-1)^{i+j} \det(\Delta_{ij}).$

Z tohto možno odvodiť nasledujúce dve tabuľky.

\begin{table}[h]
\centering
\begin{tabular}{|c|c|c|p{2.2cm}|p{2.15cm}|p{2cm}|}
\hline
Typ & $\delta$ & $s\delta$ a $T$ & $\Delta > 0$ & $\Delta < 0$ & $\Delta = 0$ \\
\hline
eliptický & $\neq 0$ & $s\delta>0$ a $T>0$ & $\emptyset$ & elipsoid & $\emptyset$ \\
\hline
hyperbolický & $\neq 0$ & $s\delta>0$ alebo $T \leq0$ & jednodielny hyperboloid & dvojdielny hyperboloid & kužeľ \\
\hline
parabolický & $=0$ & & hyperbolický paraboloid & eliptický paraboloid & valcové a reducibilné plochy \\
\hline
\end{tabular}
\caption{Klasifikácia kvadrík.}
\label{tab:classification_of_quadrics}
\end{table}

%\begin{table}[h]
%\begin{adjustbox}{center}
%\centering
%\begin{tabular}{|c|c|c|p{2.2cm}|p{2.15cm}|p{2cm}|}
%\hline
%Typ & $\delta$ & $s\delta$ a $T$ & $\Delta > 0$ & $\Delta < 0$ & $\Delta = 0$ \\
%\hline
%eliptický & $\neq 0$ & $s\delta>0$ a $T>0$ & $\emptyset$ & elipsoid & $\emptyset$ \\
%\hline
%hyperbolický & $\neq 0$ & $s\delta>0$ alebo $T \leq0$ & jednodielny hyperboloid & dvojdielny hyperboloid & kužeľ \\
%\hline
%parabolický & $=0$ & & hyperbolický paraboloid & eliptický paraboloid & valcové a reducibilné plochy \\
%\hline
%\end{tabular}
%\end{adjustbox}
%\caption{Klasifikácia kvadrík.}
%\label{tab:classification_of_quadrics}
%\end{table}

\begin{table}[h]
\centering
\begin{tabular}{|l|l|l|l|}
\hline
Typ & $T$ & $\Delta' \neq  0$ & $\Delta' = 0 $ \\
\hline
\multirow{2}{*}{eliptický} & \multirow{2}{*}{$> 0$} & ak $s\Delta < 0$, tak eliptický valec & \multirow{2}{*}{bod} \\
& & ak $s\Delta > 0$, tak $\emptyset$ &\\
\hline
hyperbolický & $< 0$ & hyperbolický valec & dve rôznobežné roviny \\
\hline
\multirow{3}{*}{parabolický} & \multirow{3}{*}{$= 0$} & \multirow{3}{*}{parabolický valec} & dve rovnobežné roviny \\
& & & rovina \\
& & & $\emptyset$ \\
\hline
\end{tabular}
\caption{Klasifikácia degenerovaných kvadrík.}
\label{tab:degenerate_quadrics}
\end{table}
\end{document}
