\chapter*{Úvod} % chapter* je necislovana kapitola
\addcontentsline{toc}{chapter}{Úvod} % rucne pridanie do obsahu
\markboth{Úvod}{Úvod} % vyriesenie hlaviciek
Pojem obálky systému kriviek po prvýkrát zaviedol v roku 1682 Ehrenfried Walther von Tschirnhaus, nemecký matematik, fyzik a lekár. Neskôr francúzsky matematik, Baron Pierre Charles François Dupin, v dvoch knižných publikáciách Développements de géométrie (Rozvoj geometrie, 1813) a Applications de géométrie et de mécanique (Aplikácie geometrie a mechaniky, 1822) opísal triedu plôch zvaných cyklidy - obálky jednoparametrických systémov guľových plôch dotýkajúcich sa danej trojice guľových plôch \cite{Ciz2017}.

Obálky plôch majú široké využitie vo viacerých odvetviach, ako optika, architektúra, robotika. V optike sa obálka používa na popis dráhy svetelných lúčov, ktoré prechádzajú cez zakrivený povrch. Výpočet obálky pohybujúcej sa plochy sa vyskytuje aj pri CNC obrábaní. \textit{Computer Numerical Control machining} je výrobný subtraktívny proces, pri ktorom počítač riadi stroje, vŕtačky, frézy a sústruhy tak, aby odstraňovali materiál z obrobku, kým nevytvoria požadovaný tvar. Pri tomto procese rezný nástroj vytvára rotačnú plochu - obálku pohybom okolo svojej osi \cite{Skop20}. Obálky sa tiež používajú na výpočet trajektórie projektilov vo vzduchu. Napríklad, pri riešení problému pohybu projektilu vrhaného pod určitým uhlom k horizontu, používame rovnicu obálky na určenie maximálneho dosahu letu projektilu. \cite{Chud09}. Medzi ďalšie aplikácie obálok patrí \textit{tollerancing} - krivka s kontrolou chyby, bezkolízne plánovanie pohybu robota, konštrukcia znakov v písmach pre typografické systémy, teda dizajn písma \cite{Pott09}. Medzi obálky patria aj kanálové plochy, rúrkové plochy a plochy používané v počítačom podporovanom geometrickom dizajne CAGD, ktoré sa vyskytujú ako zmiešavacie a prechodové plochy medzi potrubiami. Tieto plochy sú využívané aj v architektúre.

Ako príklad uvádzame Webbov most na obr. \ref{fig:webb_bridge} v Melbourne. Tvar mosta vzdáva hold histórii domorodého obyvateľstva a je vytvorený podľa tradičnej rybárskej pasce Koorie, ktorá sa používala na lov úhorov.

\begin{figure}[h]
    \centering
    \begin{subfigure}[b]{0.6\textwidth}
        \centering
        \includegraphics[width=\textwidth]{images/webbbridge.jpg}
        \caption[Webbov most.]{Webbov most \cite{WebbBridge}.}
        \label{fig:webb_bridge}
    \end{subfigure}
    \hfill
    \begin{subfigure}[b]{0.6\textwidth}
        \centering
        \includegraphics[width=\textwidth]{images/theater.jpg}
        \caption[Hudobné divadlo a výstavná sieň.]{Hudobné divadlo a výstavná sieň \cite{MusicTheater}.}
        \label{fig:theater}
    \end{subfigure}
    \caption{Projekty v architektúre.}
    \label{fig:projects}
\end{figure}

Ďalším zaujímavým architektonickým projektom je hudobné divadlo a koncertná sála v gruzínskom Tbilisi, obr. \ref{fig:theater}. Táto budova sa skladá z dvoch prepojených častí, ktoré poskytujú panoramatický výhľad na rieku a historické centrum mesta \cite{Mesz18}.

Obálku jednoparametrického systému sfér hľadáme tak, že v každom parametri systému rátame prienik sféry a derivácie systému. Ukáže sa, že prienikom je vždy kružnica a teda obálku môžeme zostrojiť použitím týchto kružníc. 

Tento koncept chceme zovšeobecniť na plochy druhého rádu, pričom prvým zovšeobecnením, ktoré sa ponúka, je obálka elipsoidov. Naša práca sa zaoberá otázkou, ako zostrojiť obálku elipsoidov, ak budú škálované faktorom $a$ v dotykovom smere krivky a v zostávajúcich dvoch smeroch budú škálované faktorom $b$, pričom $a > b.$ Teda skúmame elipsoidy, ktoré budu natiahnuté v smere krivky.

Práca je rozdelená do štyroch kapitol. V prvej kapitole spomíname prístupy výpočtu obálky v rovine a uvádzame teoretické pozadie obálky sfér potrebné na zostrojenie obálky elipsoidov. V druhej kapitole sa nachádzajú odvodené rovnice pre obálku elíps a elipsoidov a ich geometrická interpretácia. Tretia kapitola pozostáva zo špecifikácie algoritmov, vymenovania a zdôvodnenia použitého softvéru, postupu implementácie algoritmov a parametrov 3D tlače. Vo štvrtej kapitole vyhodnocujeme uvedené postupy a na záver uvádzame pár vizualizácií a časových meraní.