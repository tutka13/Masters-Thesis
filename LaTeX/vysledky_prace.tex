\chapter{Výsledky práce}
Hlavným cieľom práce bolo vypočítať obálku plôch druhého radu, kde sme za plochu druhého rádu považovali elipsoid naškálovaný v dotykovom smere krivky konštantou $a$.
Stručne predstavte účel svojho výskumu a pripomeňte čitateľom hlavné ciele. Zdôraznite význam vašich zistení v kontexte širšej oblasti výskumu.Prezentujte svoje matematické výpočty a diskutujte o výsledkoch. Popíšte implementáciu výpočtov v programe Blender Python a prediskutujte získané výsledky.
Prezentujte povrchy vytvorené v programe Blender a prediskutujte ich význam pre ciele vášho výskumu.
Prezentujte všetky 2D grafy alebo vizualizácie, ktoré dopĺňajú vaše výsledky.
Analyzujte dôsledky vašich výsledkov vo vzťahu k vašim výskumným otázkam.
Prediskutujte všetky neočakávané výsledky alebo obmedzenia, s ktorými ste sa počas práce stretli.
Porovnajte svoje výsledky s existujúcou literatúrou a predchádzajúcim výskumom v danej oblasti.
Zdôraznite akýkoľvek prínos vašej práce k rozvoju poznatkov v oblasti počítačovej grafiky a geometrie.

\section{Príklady kriviek}
\section{Príklady plôch}
\section{Budúca práca}
Môžeme uvažovať škálovanie $c$ v binormálovom smere Frenetovho repéru ku krivke $m(t)$. Namiesto konštantného škálovania použiť funkcie škálovania $a(t), b(t),$ príp. $c(t)$, napríklad funkcie normy vektorov Frenetovho repéra, $a(t) = \| \vec{t} \|, b(t) = \| \vec{n} \| $ a $c(t) = \| \vec{b} \|$. Ďalšou prácou by mohlo byť upustiť od škálovania v jednotlivých smeroch Frenetovej bázy a škálovať jednoparametrocký systém elipsoidov ľubovoľne. Potom prejsť všeobecne k plochám druhého rádu.
Taktiež sa dá dokončiť skript do tvaru elipsy a kružnice, kde by bolo potrebné riešiť otázku ako vygenerovať mesh a teda pospájať objekty v scéne vhodným spôsobom.