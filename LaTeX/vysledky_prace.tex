\chapter{Výsledky práce}
Hlavným cieľom práce bolo vypočítať obálku plôch druhého radu, kde sme za plochu druhého rádu považovali elipsoid naškálovaný v dotykovom smere krivky konštantou $a$. Obálku jednoparametrického systému elipsoidov sme počítali rovnakým prístupom ako obálku sfér, a to tak, že sme v každom parameteri $t$ systému zistili prienik systému s jeho deriváciou. Systém ležal na krivke $m(t).$
Zistili sme, že v každom parametri $t$ je prienikom systémov kružnica a v tých parametroch $t,$ pre ktoré platí
$$
\frac{b}{a^2-b^2} > \kappa(t),
$$  
kde $\kappa(t)$ je krivosť krivky $m(t)$ je prienikom systémov aj elipsa.
Tento výpočet sme implementovali v programe Blender pomocu Blender Python API a tak sme zostrojili menšiu databázu jednoparametrických systémov sfér, jednoparametrických systémov elipsoidov a ich obálok. V tejto časti uvedieme pár vizualizácií, kde uvedieme parametre plôch.

Pri konštrukcií plôch z elipsoidov sme vyriešili o rozmer menší prípad a to obálku elíps v rovine. Pre tie sme výsledky vizualizovali v programe Desmos, kde sme zadávali matematické výrazy vypočítané v Pythone. 

Popíšte implementáciu výpočtov v programe Blender Python a prediskutujte získané výsledky.
Prezentujte povrchy vytvorené v programe Blender a prediskutujte ich význam pre ciele vášho výskumu.
Prezentujte všetky 2D grafy alebo vizualizácie, ktoré dopĺňajú vaše výsledky.
Analyzujte dôsledky vašich výsledkov vo vzťahu k vašim výskumným otázkam.
Prediskutujte všetky neočakávané výsledky alebo obmedzenia, s ktorými ste sa počas práce stretli.
Porovnajte svoje výsledky s existujúcou literatúrou a predchádzajúcim výskumom v danej oblasti.
Zdôraznite akýkoľvek prínos vašej práce k rozvoju poznatkov v oblasti počítačovej grafiky a geometrie.

\section{Príklady kriviek}
\section{Príklady plôch}
\section{Budúca práca}
Môžeme uvažovať škálovanie $c$ v binormálovom smere Frenetovho repéru ku krivke $m(t)$. Namiesto konštantného škálovania použiť funkcie škálovania $a(t), b(t),$ príp. $c(t)$, napríklad funkcie normy vektorov Frenetovho repéra, $a(t) = \| \vec{t} \|, b(t) = \| \vec{n} \| $ a $c(t) = \| \vec{b} \|$. Ďalšou prácou by mohlo byť upustiť od škálovania v jednotlivých smeroch Frenetovej bázy a škálovať jednoparametrocký systém elipsoidov ľubovoľne. Potom prejsť všeobecne k plochám druhého rádu.
Taktiež sa dá dokončiť skript do tvaru elipsy a kružnice, kde by bolo potrebné riešiť otázku ako vygenerovať mesh a teda pospájať objekty v scéne vhodným spôsobom.