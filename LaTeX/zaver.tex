\chapter*{Záver}
\addcontentsline{toc}{chapter}{Záver}
Hlavným cieľom práce bolo vypočítať obálku jednoparametrického systému plôch druhého stupňa $Q$, kde sme za plochu druhého stupňa považovali elipsoid naškálovaný v dotykovom smere krivky $m(t)$ konštantou $a$ a v zostávajúcich smeroch krivky konštantou $b$. Obálku jednoparametrického systému elipsoidov $Q$ sme počítali rovnakým prístupom ako obálku jednoparametrického systému sfér $S$, a to tak, že sme v každom parameteri $t$ systému $Q$ zistili prienik systému s jeho deriváciou $\dot{Q}$. Systém ležal na krivke $m(t),$ kde $t \in I \subset \mathbb{R}.$

Zistili sme, že určujúcim faktorom je krivosť krivky $\kappa(t)$ a pomer škálovacích koeficientov $\lambda = \big| \frac{b}{a^2 - b^2} \big|.$ Dokázali sme, že v každom parametri $t$ je prienikom systémov $Q$ a $\dot{Q}$ charakteristická kružnica $c_t$ a v tých parametroch $t,$ pre ktoré platí
$$
\kappa(t) > \lambda,
$$  
do prieniku systémov prispieva aj charakteristická elipsa $e_t$.

Obálky sfér a elipsoidov pre vybrané parametrizácie kriviek sme zostrojili v programe Blender pomocu Blender Python API. 

Pre 
$$
\kappa(t) \leq \lambda,
$$  
sme v $t$ vykreslili charakteristickú kružnicu $c_t$ a pre
$$
\kappa(t) > \lambda,
$$  
sme v $t$ vykreslili pôvodný elipsoid zo systému.
Potom sme charakteristické kružnice pospájali nástrojom programu Blender, ktorý automaticky vygeneroval pletivo. 

Týmto spôsobom sme zostrojili menšiu databázu jednoparametrických systémov sfér, jednoparametrických systémov elipsoidov a ich obálok, ktorá sa nachádza spolu s ďalšími súbormi na webovej lokalite v GitHub repozitári \url{https://github.com/tutka13/Masters-Thesis/tree/main}. 

13 vybraných príkladov plôch sme vytlačili v 3D tlačiarni.

V prvej kapitole sme uviedli prehľad obálok v rovine, prístupy výpočtu a tiež odvodenie rovníc pre obálku systému sfér. V druhej kapitole sme odvodili rovnice obálky systému elipsoidov. Tretia kapitola pozostáva zo špecifikácie algoritmov, vymenovania a zdôvodnenia použitého softvéru, postupu implementácie a parametrov 3D tlače. Vo štvrtej kapitole sme ilustrovali výsledky práce na konkrétnych príkladoch, spomenuli sme ťažkosti v oblasti 3D tlače a identifikovali sme otvorené problémy. Taktiež sme naznačili možnosti budúcich výskumných smerov.